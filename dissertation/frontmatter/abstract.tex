In the magnetospheres of neutron stars, as well as in the accretion disk coronae and black hole jets, energy is mainly stored in the form of a magnetic field. We observe these objects in a wide range of frequencies, from radio to TeV, and their typically high luminosities suggest that a considerable fraction of the magnetic energy is extracted via magnetic reconnection and converted into the observed radiation. In certain objects the energies are so extreme that the emitted light can backreact on plasma via quantum electrodynamic (QED) processes, significantly altering the dynamics of reconnection. This thesis is focused on studying how exactly reconnection operates in these extreme environments. 

Because of the complexity of these systems, we typically study their dynamics numerically. In the first chapter of my thesis I describe novel numerical algorithms to self-consistently model the interaction between plasma and light in the ultra-relativistic regime. In particular, I present techniques that allow to include QED processes, such as radiation reaction, Compton scattering, and pair-production/-annihilation in plasma simulations. In the next chapter I describe the physics behind the new particle acceleration mechanism discovered during numerical simulations of reconnecting current sheets. This mechanism allows to overcome the maximum energy barrier in reconnection by slowly accelerating particles in continuously compressing magnetic islands, which were previously thought to be unimportant in this context. These results potentially explain the broken power-law radiation spectrum observed in some blazars. I then present global simulations of neutron star magnetospheres and study the 3D reconnection dynamics responsible for tapping the large magnetic field energies in these objects. I demonstrate how microscopic plasma instabilities onset and evolve in large-scale reconnecting current layers in the magnetospheres of young energetic pulsars, and how the gamma-ray emission emerges from this process. This emission is so abundant and energetic that it can produce matter (pair plasma) which is fed back to the magnetosphere and strongly modifies the reconnection process. In the final chapter I study this effect from first principles in localized simulations. I demonstrate that this pair-production process has a negative feedback on reconnection, reducing its particle acceleration efficiency. This effect explains the observed universality of high-energy cutoff frequency in gamma-ray pulsars. 