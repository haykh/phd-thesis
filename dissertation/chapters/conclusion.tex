\chapter{Conclusion}
\label{conclusion}

Relativistic magnetic reconnection is a fast and efficient process of magnetic field dissipation, during which the energy stored in magnetic fields is extracted and deposited into plasma particles. Owing to its extremely violent character and the ability to accelerate particles to form non-thermal distributions, this process is thought to power many of the persistent and transient emission characteristics in compact astrophysical objects: from the magnetospheres of neutron stars and black holes to accretion disk coronae and jets. In many of these objects reconnection dynamics is also largely affected by the matter-light interaction, which may strongly alter their observational appearance. 

Up until now reconnection has been modeled using kinetic plasma simulations which mostly ignored these radiative effects. In \S\ref{ch:numerics} of this thesis I presented novel computational methods for including radiative and quantum electrodynamic (QED) effects in particle-in-cell plasma simulations. These techniques allow to self-consistently model plasma processes in extreme astrophysical conditions, where radiative reaction force is dynamically important, and the emitted light is strongly coupled to plasma via Compton scattering, two-photon pair production and $e^\pm$ pair-annihilation. I also proposed several simplified numerical experiments to test the performance of these algorithms and compare the outcome with the analytic predictions. 

Reconnection can accelerate particles to ultra-relativistic energies generating a non-thermal power-law distribution. The maximum energy of this distribution, as well as the power-law index are thought to be uniquely controlled by the so-called magnetization parameter in the upstream unreconnected plasma. In \S\ref{ch:plasmoids}, I explored a new powerful acceleration channel during magnetic reconnection that allows to overcome the previously known limit on the maximum energy, and form a broken power-law. This acceleration mechanism operates in continuously compressing magnetic islands, called plasmoids, that form in the non-linear stage of the reconnection and carry away the reconnected flux and energized particles. Theoretical model of this process allowed me to extrapolate its implications to real astrophysical systems. This mechanism may explain the observed broken power-law distributions and anomalously high energies in some of the blazar jets.

Macroscopic current layers are thought to emerge in the magnetospheres of neutron stars as a result of the large-scale magnetic topology. Magnetic reconnection in these layers may dissipate a significant amount of Poynting flux carrying away the rotational power of the neutron star. In \S\ref{ch:pulsar}, I recreated the magnetospheres of neutron stars in PIC simulations and from first principles reproduced the reconnection plasma dynamics in the current layer. I demonstrated that microscopic plasma physics in the thin reconnecting layer controls the overall energy dissipation rate in the entire magnetosphere. The fraction of energy dissipated is uniquely set by the rate of magnetic reconnection, and appears to be insensitive to synchrotron radiation efficiency and bulk motion of the magnetospheric wind. I have also studied the acceleration of particles during reconnection and resulting synchrotron spectra for pulsars in different radiative regimes. I provided a possible explanation for why pulsars with higher spin-down power and stronger synchrotron cooling typically exhibit $\gamma$-ray spectra that peak at lower energies. 

In the most energetic $\gamma$-ray pulsars, such as the Crab, synchrotron emission in the current layer is so abundant and energetic that the emitted photons can interact with each other producing secondary $e^\pm$ pairs. In \S\ref{ch:pairproduction}, I studied this process by self-consistently coupling photon emission and two-photon pair production process to plasma simulations of the magnetically reconnecting current layer. In certain regimes pair production is so abundant that the current layer is dominated by the pairs produced \emph{in-situ}. This process feeds the radiated energy back into the current sheet in the form of secondary pairs, inhibiting the acceleration efficiency of reconnection. As a result, a negative feedback loop emerges, which controls the overall particle acceleration efficiency. This effect explains the universality of $\gamma$-ray cutoff energies in young pulsars, which is around few to $10$ GeV and is surprisingly insensitive to the strength of the magnetic field in the reconnecting current layer.

\subsection*{\small \it Future prospects}

Large-scale reconnecting current layers can be intermittently produced in \emph{the magnetically arrested accretion disks} of supermassive black holes (such as the one in M87; see, e.g., \citealt{2020ApJ...900..100R,ripperdainprep}). Parameter regimes in these layers are close to those in $\gamma$-ray pulsars: the synchrotron cooling efficiency is marginal and the two-photon pair production process is abundant. These transient reconnection events most likely power the observed TeV flares \citep[see, e.g.,][]{2006Sci...314.1424A} by upscattering soft background photons on pair plasma energized during reconnection. Modeling radiative reconnection in this regime to reproduce the emerging TeV emission, as well as the lower energy counterpart, is the next logical step in this topic. 

In the hard states of X-ray binaries millisecond-duration $\gamma$-ray flares \citep[see, e.g.,][]{2003MNRAS.343L..84G} are thought to be powered by transient reconnection events from disrupting magnetic flux ropes in \emph{the accretion disk coronae} \citep{1979ApJ...229..318G,2008ApJ...682..608U,2017ApJ...850..141B}. In these environments, synchrotron and inverse Compton cooling are very efficient, while the optical depth of the emission (which is close to unity) is self-consistently controlled by the balance between two-photon pair production and electron-positron annihilation. Dynamics of the reconnection has never been studied in such an exotic regime.

In \emph{magnetars}, non-linear Alfv\'enic fluctuations generated at the surface of the star may disrupt the magnetosphere, producing large-scale reconnecting current sheets \citep{2020arXiv201107310B,2020ApJ...900L..21Y}. Emission from these current sheets may be responsible for the observed $\gamma$-ray flares and short-duration X-ray bursts \citep{2017ARA&A..55..261K}. Simulations show that similar dynamics may occur in the coupled magnetospheres of \emph{the coalescing binary neutron stars} shortly before they merge \citep{2020ApJ...893L...6M}. Since the parameter regimes are very close in these systems, this transient would generate a strong high-energy precursor to the neutron star merger event. In the regimes applicable both to magnetar flares and binary neutron stars, pair-production and annihilation are self-balanced, while the Compton scattering is so violent that it can effectively couple plasma to radiation, providing an effective ``photon viscosity'' and possibly strongly inhibiting the reconnection process. 

In this thesis I described a novel set of tools and techniques for modeling plasma kinetics (such as reconnection, turbulence, collisionless shocks, etc) mediated by radiative and QED processes. This extreme regime has largely been beyond the reach of first-principle plasma simulations, and many of the observational implications have been overlooked. My thesis is intended to serve as one of the first steps towards enabling us to reproduce and understanding the microscopic plasma behavior in these violent environments. Above I presented just a few of the examples of systems where radiative and QED processes may play a vital role both in terms of the fundamental plasma physics, and also in terms of interpreting the observations from compact sources.
