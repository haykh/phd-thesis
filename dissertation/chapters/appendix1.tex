\chapter{}
\section{Cross sections for the QED processes}
\label{appendix:num-crosssections}

In this appendix we present the expressions for the differential and total cross sections of three different QED processes included in \texttt{Tristan-MP v2} and described in section~\ref{sec:num-QED}. All the expressions are taken from the book by \cite{1985quel.book.....A}. In this section $\sigma_T$ is the Thomson cross section, $d\Omega$ is the differential solid angle, and $\theta$ is the scattering angle between the initial and final momenta; all the energies are normalized to $m_e c^2$. Also notice, that the total cross section is a relativistic invariant.

For \emph{Compton scattering} of a photon on an electron (or positron) it is convenient to express the cross sections in the $e^\pm$ rest-frame:
\begin{equation}
    \begin{aligned}
        \frac{d \sigma_C}{d\Omega} &= \frac{3}{16\pi} \sigma_T 
            \frac{1 + \cos^2{\theta}}{\left(1 + \varepsilon_1(1-\cos{\theta})\right)^2}
            \left(1 + \frac{\varepsilon_1^2(1-\cos{\theta})^2}{(1+\cos^2{\theta})\left(1 + \varepsilon_1 (1-\cos{\theta})\right)}\right),\\
        \sigma_C &= \frac{3}{4}\sigma_T\left[
            \frac{1+\varepsilon_1}{\varepsilon_1^3}
                \left( \frac{2\varepsilon_1(1+\varepsilon_1)}{1+2\varepsilon_1} - \ln{(1 + 2\varepsilon_1)}\right)
            + \frac{\ln{(1+2\varepsilon_1)}}{2\varepsilon_1} - \frac{1 + 3\varepsilon_1}{(1 + 2\varepsilon_1)^2},
        \right]
    \end{aligned}
\end{equation}

\noindent where $\varepsilon_1$ is the energy of the photon before the scattering (in the  $e^\pm$ rest-frame).

For \emph{two-photon pair production} process the cross sections are given in the center-of-momentum frame of the two interacting photons:

\begin{equation}
    \begin{aligned}
        \frac{d \sigma_{\gamma\gamma}}{d\Omega} &= \frac{3}{32\pi} \sigma_T 
            \frac{\sqrt{\varepsilon_0^2-1}}{\varepsilon_0^3}
            \left[
                \frac{2 \varepsilon_0^2 -1 + (\varepsilon_0^2 - 1)\sin^2{\theta}}{\cos^2{\theta} + \varepsilon_0^2\sin^2{\theta}}
                - \frac{2(\varepsilon_0^2 - 1)^2\sin^4{\theta}}{\left(\cos^2{\theta}+\varepsilon_0^2 \sin^2{\theta}\right)^2}
            \right] \\
        \sigma_{\gamma\gamma} &= \frac{3}{8}\sigma_T \frac{1}{\varepsilon_0^2}\left[
            \left(2 + \frac{2}{\varepsilon_0^2} - \frac{1}{\varepsilon_0^4}\right)\ln{\left(\varepsilon_0 + \sqrt{\varepsilon_0^2 - 1}\right)} - 
            \sqrt{1 - \frac{1}{\varepsilon_0^2}}\left(1 + \frac{1}{\varepsilon_0^2}\right)
        \right],
    \end{aligned}
\end{equation}

\noindent where $\varepsilon_0$ is the energy of either of the photons in the center-of-momentum frame. For two photons with energies $\varepsilon_1$ and $\varepsilon_2$ in an arbitrary frame with an incident angle $\phi$ w.r.t. each other we find: 
\begin{equation}
    \varepsilon_0 = \frac{1}{2}\frac{\varepsilon_1 \varepsilon_2}{(m_e c^2)^2}\left(1-\cos{\phi}\right).
\end{equation}

\noindent Notice, that this interaction is prohibited if $\varepsilon_0 < 1$, i.e., if the total energy of interacting photons in the center-of-momentum frame is less than $2m_e c^2$.

For \emph{pair-annihilation} the cross sections are also given in the center-of-momentum frame of interacting electron and positron:

\begin{equation}
    \begin{aligned}
        \frac{d \sigma_{\rm ann}}{d\Omega} &= \frac{3}{32\pi} \sigma_T \frac{1}{v_0 \varepsilon_0^2}
        \frac{1+v_0^2\sin^2{2\theta} - v_0^4(1 - \sin^4{\theta})}{\left(1 - v_0^2\cos^2{\theta}\right)^2 },\\
        \sigma_{\rm ann} &= \frac{3}{32}\sigma_T \frac{1}{v_0\varepsilon_0^2}
        \left(
            \frac{3-v_0^4}{v_0} \ln{\left|\frac{1+v_0}{1-v_0}\right| + 2(v_0^2 - 2)}
        \right)
        ,
    \end{aligned}
\end{equation}

\noindent with $\varepsilon_0$ and $v_0$ being, correspondingly, the energy and the magnitude of the three-velocity of either of the interacting particles in the center-of-momentum frame.